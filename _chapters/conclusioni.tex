\chapter{Conclusioni}
L'elaborazione digitale di segnali è un argomento vastissimo, con varie sfaccettature e varianti: una cosa che funziona molto bene in un caso, potrebbe funzionare molto male in un altro caso e tutto dipende a cosa bisogna applicare tale operazione. Un esempio lampante è il tempo di elaborazione della FFT presentate nell'ultimo capitolo: la CPU ha prestazioni eccezionali per pochi punti, mentre la GPU ha prestazioni incredibili nel caso opposto. Questo non significa che l'utilizzo di una o l'altra piattaforma sia esclusiva, anzi, per come è architetturata ora la tecnologia il caso ottimo è quello di utilizzare entrambi i ``centri di calcolo''.

La sfida rimane quella di modificare gli algoritmi preesistenti in una loro versione che possa essere calcolata in modo parallelo su una GPU. Come si è visto, non sempre è possibile parallelizzare l'intero processo in una volta sola e a volte bisogna scendere a compromessi e utilizzare funzioni di sincronizzazione. 

Nonostante ciò, le prestazioni ottenute dalla GPU sono a dir poco incredibili e soprattutto i grafici presentati mostrano uno \textit{scaling} molto promettente. Non a caso, infatti, questi dispositivi vengono utilizzati per la ricerca nel campo delle intelligenze artificiali e nella costruzione di supercomputer.