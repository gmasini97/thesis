\chapter{Introduzione}
La disciplina denominata \textit{DSP}, dall'inglese \textit{Digital Signal Processing} ovvero ``elaborazione di segnali digitali'', è stata ed è tuttora parte fondamentale dello sviluppo tecnologico digitale che caratterizza la storia dell'umanità a partire dalla seconda metà del ventesimo secolo. Ciò è dovuto al fatto che gran parte delle grandezze di interesse scentifico e ingegneristico che debbono essere analizzate e processate dai calcolatori hanno natura di segnali, i quali necessitano di particolari accortezze e algoritmi nella loro elaborazione.

\section{Scopo}
La presente tesi si pone come obbiettivo quello di studiare e implementare alcuni dei tanti algoritmi che vengono utilizzati nell'ambito DSP sia dal punto di vista della CPU, sia una loro possibile parallelizzazione su GP-GPU. Ciò non significa che gli algoritmi riportati siano nella loro forma più efficiente, bensì sono mostrati in modo da far risaltare le differenze implementative che espongono sulle due piattaforme. Inoltre gli algoritmi vengono realizzati con un minimo utilizzo di librerie esterne, in quanto è nell'interesse della tesi e dell'autore l'approfondimento degli algoritmi stessi e lo studio del loro funzionamento interno.



\section{Inquadramento}
[divisione sezioni tesi oppure cosa fanno altri/progetti correlati]