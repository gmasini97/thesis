\chapter{Introduzione}
La disciplina denominata \textit{DSP}, dall'inglese \textit{Digital Signal Processing} ovvero ``elaborazione digitale di segnali'', è stata ed è tuttora parte fondamentale dello sviluppo tecnologico digitale che caratterizza la storia dell'umanità a partire dalla seconda metà del ventesimo secolo. Ciò è dovuto al fatto che gran parte delle grandezze di interesse scentifico e ingegneristico che debbono essere analizzate ed elaborate hanno natura di segnali, i quali necessitano di particolari accortezze e algoritmi per essere elaborati da un dispositivo a capacità di calcolo e memoria limitate, come il calcolatore elettronico.

L'elaborazione digitale dei segnali interessa particolarmente il campo delle telecomunicazioni, dove si lotta per ottenere bitrate sempre maggiori su lunghissime distanze. Un esempio lampante sono le linee telefoniche: dai commutatori analogici, costosi e poco pratici, si è passati ai canali digitali, che a parità di qualità audio offrono un maggior numero di connessioni contemporanee sullo stesso supporto (il doppino telefonico), non necessitano di interruttori analogici e soprattutto hanno un costo sia in termini di costruzione sia di messa in operazione e manutenzione nettamente minore.

Non bisogna però restringere il proprio campo visivo alle sole telecomunicazioni, poiché le tecnologie DSP vengono largamente utilizzate anche in altri campi e risultano fondamentali per applicazioni come video e audio processing, applicazioni mediche, militari, finanziarie e di ricerca.

\section{Scopo}
La presente tesi si pone come obbiettivo quello di studiare e implementare alcuni dei tanti algoritmi che vengono utilizzati nell'ambito DSP sia dal punto di vista essenzialmente sequenziale del processore, sia una loro possibile parallelizzazione su scheda grafica. Ciò non significa che gli algoritmi riportati siano nella loro forma più efficiente o performante, bensì sono mostrati in modo da far risaltare le differenze implementative che espongono sulle due piattaforme. Inoltre gli algoritmi vengono realizzati con un minimo utilizzo di librerie esterne, in quanto è nell'interesse della tesi e dell'autore l'approfondimento degli algoritmi stessi e lo studio del loro funzionamento interno.

Per portare a termine tale scopo è stata necessaria la stesura di un programma in grado di eseguire gli algoritmi studiati e implementati sia su CPU sia su GPU. Si è deciso di limitarsi all'elaborazione di file audio a canale singolo poiché oltre ad ottenere un riscontro in termini di forma d'onda e spettri di frequenza e fase è possibile anche verificarne il funzionamento in base all'effetto che si ottiene nell'ascolto del risultato. Il programma, con le dovute modifiche, si può estendere anche all'elaborazione di segnali diversi dall'audio.

\section{Inquadramento}
Il presente elaborato espone nel capitolo \ref{cap:nozioni} alcune nozioni matematiche necessarie per la corretta comprensione e spiegazione delle implementazioni degli algoritmi presentati in seguito. Viene spiegato, inoltre, come si utilizza la GPU dal punto di vista implementativo, quali sono i ``componenti di calcolo'' principali e come vi si interfaccia con le API di CUDA.

Nel capitolo \ref{cap:implementazione} vengono presentate e spiegate le parti più interessanti degli algoritmi studiati, i quali sono interamente disponibili nel repository GitHub della tesi \cite{repo}.